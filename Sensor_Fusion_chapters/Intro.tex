\section{Introduction}
Proton exchange membrane fuel cell (PEMFC) is considered one of the hydrogen energy consumption terminals in the automotive field due to its zero-emission, high energy density, and low operating temperature \cite{zhuHighprecisionIdentificationPolarization2022,maResearchElectrochemicalImpedance2021}. However, reliability and durability still need to be further improved\cite{haslamAssessingFuelCell2012}. Water management is one of the key approaches to enhance the durability of PEMFC. In general, the normal operation of PEMFC requires that the membrane contains the appropriate amount of water. Both dryness and flooding can cause negative impacts on the durability of PEMFC\cite{yangInplaneTransportEffects2012,owejanVoltageInstabilitySimulated2007,endohDegradationStudyMEA2004,kadykNonlinearFrequencyResponse2012}. Meanwhile, the internal water state of the PEMFC is dynamically changed during operation. Thus, it is necessary to identify the internal water state of the PEMFC accurately and quickly, and control it within a reasonable range. However, due to the difficulty in directly measuring the water content of the fuel cell during actual operation, indirect online methods need to be used for observation \cite{damourPolymerElectrolyteMembrane2015,shinOnlineWaterContents2020}.
\par
The online observation methods for the water content of PEMFC can be divided into feature-based methods \cite{kimStateofhealthDiagnosisBased2012,maWaterContentDiagnosis2020,xuNonlinearObservationInternal2017}, data-driven methods \cite{zhangIntelligentSimultaneousFault2019,baoModelingControlAir2006,zhangUnscentedKalmanFilter2020}, and mechanism model methods\cite{linReviewHydrogenFuel2019,benmounaFaultDiagnosisMethods2017}. Among them, the feature-based methods achieve fault diagnosis by extracting features sensitive to faults from external observations \cite{zhangUnscentedKalmanFilter2020}. But the feature-based method lacks analysis of the relationship between the extracted features and the internal state, and cannot quantify the internal state of the fuel cell \cite{damourPolymerElectrolyteMembrane2015}. The data-driven methods obtain models by learning and mining the rules in experimental data. It is suitable for online monitoring and fault diagnosis because of the large amount of calculation in training and the small amount of calculation in practical application \cite{shinOnlineWaterContents2020,vichardHybridFuelCell2021,cooperElectricalTestMethods2006}. While the data-driven methods' model quality is greatly affected by experimental data, it is difficult to obtain accurate models \cite{benaggouneDatadrivenMethodMultistepahead2022,liDatadrivenFrameworkPerformance2022,liMultiobjectiveOptimizationDataDriven2021}. Moreover, the data-driven methods, which require measurable inputs and outputs for training, are difficult to use for estimating the states that cannot be directly measured and quantifying the internal water content\cite{meraghniDatadrivenDigitaltwinPrognostics2021}. Mechanism model methods have a clear physical meaning and can quantify the internal state of PEMFC \cite{yangInplaneTransportEffects2012}. Although part of the mechanism of the PEMFC is still unclear, and the effect of the mechanism model is not ideal in most cases, the estimation method based on the mechanism model is still the best solution for the internal state which is difficult to measure \cite{benmounaFaultDiagnosisMethods2017,liInvestigationMeasurementUncertainty2024,fullerWaterThermalManagement1993}. As the Computational Fluid Dynamics (CFD) model is a multi-dimensional and multi-phase model, it is typically used for off-line analysis. Thus, the internal state of the PEMFC can be estimated online by simplifying the mechanism model \cite{baoTwodimensionalModelingPolymer2015,baoTwodimensionalModelingPolymer2015,baoTwodimensionalModelingPolymer2015a}. However, the existing simplified mechanism models do not distinguish the flow channel, the gas diffusion layer (GDL), and the catalyst layer (CL). Meanwhile, the water content in the ionomer, the saturation of liquid water and the pressure of water vapor are rarely considered \cite{hinatsuWaterUptakePerfluorosulfonic1994}.
\par
In addition, affected by factors such as the inaccuracy of the simplified mechanism model and actual system interference \cite{huThreeDimensionalTwo2004}, the fuel cell water state estimation error is large. Moreover, when the error cannot be effectively corrected, the error can gradually accumulate as the calculation time increases\cite{liuPrognosticsMethodsDegradation2020}. Using a state observer, such errors can be corrected as new feedback is generated, therefore preventing the error accumulations. Thus, based on the simplified mechanism model, the internal state estimation can be ideally achieved by combining the observer algorithm \cite{yuanModelbasedObserversInternal2020a}.
\par
In early research, Zhang et al.\cite{zhangUnscentedKalmanFilter2020} conducted the online estimation of the liquid water content in the cathode channel and GDL based on the serial unscented Kalman filter framework, introducing the impact of cathode liquid water into the PEMFC system model, using different time scales between the channel and GDL modules, which is decoupled into sub-modules. Xu et al.\cite{xuNonlinearObservationInternal2017} established a high-order sliding mode observer considering the water transfer process and phase transition process, as well as the effect of liquid water saturation on voltage in cathode GDL\cite{moradinafchiAnionExchangeMembrane2024}. However, most state observers use voltage as the measured value rarely introducing impedance information, which has limited correction effects\cite{farcasAdaptiveControlMembrane2014}. Moreover, the Kalman filter algorithm can lose some precision when dealing with nonlinear state estimation and the particle filter has great advantages in solving nonlinear problems \cite{ParticleFiltering,vandermerweSquarerootUnscentedKalman2001}.
\par
Thus, in this work, a simplified mechanism model of PEM containing water content in ionomer, liquid water, and water vapor was established. According to the simulation and experiment, the simplified mechanism model was verified. Then, the influence of measurement noise and process noise set values on the performance of the observer was analyzed. Finally, an internal state observer based on the model and the particle filter algorithm was developed. Based on the simulation, the internal water state trend of the PEMFC was analyzed, and the performance of the state observer based on voltage, high frequency resistance (HFR), and sensor fusion was compared.