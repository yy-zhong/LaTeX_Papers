% \documentclass[preprint]{elsarticle}
\documentclass[utf8,a4paper,12pt] {ctexart}


% \usepackage{caption}
% \biboptions{sort&compress}
% \setmainfont{Times New Roman}

\begin{document}
% \CJKfamily{zhkai} 中文字体(楷书) \CJKfamily{zhyou} 中文字体(幼圆) \CJKfamily{zhyahei} 中文字体(微软雅黑)\\
% \setCJKmonofont   { FangSong }
% \journal{TBD}
不同尺寸下的电堆,极化曲线基本相同,但是不同的气体流量会对电极处液态水的量有较大影响(小尺寸电堆对于气体流量更不敏感)\cite{bonnetDesign80kWePEM2008}

\par

电堆中Cell数量的多寡会影响性能,也会导致温度不均衡,进而导致水存量的不均衡\cite{millerReviewPolymerElectrolyte2011}

\par

解释了PEMFC中水管理问题,提及了大型电堆可能与小型电堆的水生成机理有区别\cite{jiReviewWaterManagement2009}

大型电堆通常在低功率密度下工作,这可能会对水生成带来一定的影响\cite{shojayianSimulationCathodeCatalyst2024}

\bibliographystyle{elsarticle-num}
\bibliography{ref.bib}
\end{document}