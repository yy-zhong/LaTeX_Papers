\section{Conclusion}
Different conditions and the absolute value of the water content inside the fuel cell system.
\par
This study investigates the water content state of the high-power fuel cell system and constructs a water balance model. Initially, to authenticate this model, a tail gas water content detection apparatus centered on condensation was designed and established, based on the original high-power fuel cell system. Subsequently, control variable experiments involving three factors, namely, working temperature, air metering ratio, and load current, were executed to measure the water flow rate emanating from the air side and hydrogen side of the fuel cell system during steady-state operation. The experimental outcomes are as follows:
\begin{description}
    \item [1] Under larger air metering ratios, elevated coolant inlet temperatures, and higher load currents, the water content in the exhaust gas on the air side is higher; conversely, under smaller air metering ratios, lower coolant inlet temperatures, and higher load currents, the water content in the exhaust gas on the hydrogen side is higher.
    \item [2] Under the same load current, as the working temperature and air metering ratio escalate, the rate of change of the water content inside the fuel cell system progressively decreases. However, the degree of influence of working temperature and air metering ratio on the rate of change of the water content inside the fuel cell system slightly varies. When the load current is 120A and 210A, the influence of working temperature is more pronounced; whereas, when the load current is 300A, the influence of the air metering ratio is more evident.
    \item [3] With the amplification of load current, the rate of change of the water content inside the fuel cell system somewhat increases, but the increase is not substantial. Simultaneously, the range of the rate of change of the internal water content at 210A is greater than that at 120A and 300A, and the corresponding fuel cell system cannot operate stably under the conditions of 210A/55$^{\circ}C$/1.6 and 210A/55$^{\circ}C$/1.8. This further substantiates the correlation between the rate of change of the internal water content of the fuel cell system and the water content fault.
\end{description}